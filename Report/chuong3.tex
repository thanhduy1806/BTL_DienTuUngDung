\chapter{Tính toán mô phỏng}
\section{Mô hình loadcell}

Mô hình load cell được minh họa trong Hình~\ref{fig:loadcell_model}, ta có được $R_{in}$ và $R_{out}$ như sau:

\begin{figure}[H]
    \centering
    \begin{minipage}{0.35\textwidth}
        \centering
        \includegraphics[width=\textwidth]{image/Rin.png}
        \caption{$R_{in}$}
        \label{fig:rin}
    \end{minipage}
    \hfill
    \begin{minipage}{0.35\textwidth}
        \centering
        \includegraphics[width=\textwidth]{image/Rout.png}
        \caption{$R_{out}$}
        \label{fig:rout}
    \end{minipage}
\end{figure}

\begin{figure}[H]
    \centering
    \includegraphics[width=0.25\textwidth]{image/Loadcell_model_2.png}
    \caption{Mô hình Load cell mô phỏng}
    \label{fig:loadcell_model_2}
\end{figure}
\noindent
Ta có:
\[
\left\{
\begin{aligned}
R_{in} &= \dfrac{(R_0+R_f) + (R_0-R_f)}{2} + 2R_s \\
R_{out} &= R_0 - \dfrac{R_f^2}{R_0 + 2R_s}
\end{aligned}
\right.
\]
\noindent
Khi chưa có tải $m = 0$, cầu cân bằng:
\[
\rightarrow
\left\{
\begin{aligned}
R_f &= 0 \\[4pt]
R_0 &= R_{out} = 1000\,\Omega \\[4pt]
R_s &= 33\,\Omega
\end{aligned}
\right.
\]
\noindent
Khi có tải $m \neq 0 \rightarrow R_f \neq 0$. Ta có:
\[
\left\{
\begin{aligned}
I_2 &= I_3 = \dfrac{I_1}{2} \\
V_{out} &= V_a - V_b = (I_1R_s+I_3(R_0+R_f)) -(I_1R_s+I_2(R_0-R_f)) \\
\end{aligned}
\right.
\]
\begin{align*}
&\rightarrow\; V_{\text{out}} 
= \left[ I_1 R_s + \dfrac{I_1}{2}(R_0 + R_f) \right] 
- \left[ I_1 R_s + \dfrac{I_1}{2}(R_0 - R_f) \right] 
= I_1 R_f 
= R_f \dfrac{V_{\text{in}}}{R_{\text{in}}} \\
%
&\rightarrow\; R_f 
= R_{\text{in}} \dfrac{V_{\text{out}}}{V_{\text{in}}} 
= R_{\text{in}} \times V_{RO} \times \dfrac{\text{m}}{1\,\text{kg}} 
= 1066 \times 1 \times 10^{-3} \times m \\
%
&\boxed{\rightarrow\; R_f = 1.066 \times m \quad (\Omega)}
\end{align*}

Mô phỏng trên NI multisim:
%Cho hình vào đây
\begin{figure}[H]
    \centering
    \begin{minipage}{0.4\textwidth}
        \centering
        \includegraphics[width=\textwidth]{image/m=0.png}
        \caption{Load cell khi $m = 0$ kg}
        \label{fig:m=0}
    \end{minipage}
    \hfill
    \begin{minipage}{0.4\textwidth}
        \centering
        \includegraphics[width=\textwidth]{image/m=1.png}
        \caption{Load cell khi $m = 1$ kg}
        \label{fig:m=1}
    \end{minipage}
\end{figure}

\textbf{Nhận xét:} Tín hiệu ngõ ra của load cell mô phỏng đã đúng với các tính toán.

Như đã đề cập ở chương 2, ngõ ra cần khuếch đại lên 5V. Chính vì thế độ lợi là 1000. 

Ta có độ lợi của INA333 $G = 1+\frac{100k\Omega}{R_G} \rightarrow R_G = 100 \Omega$

\begin{figure}[H]
    \centering
    \includegraphics[width=0.9\textwidth]{image/loadcell_ampli.png}
    \caption{Load cell sau khi khuếch đại}
    \label{fig:loadcell_amplifier}
\end{figure}

\textbf{Nhận xét:} Tín hiệu ngõ ra của load cell sau khi khuếch đại đã giống với tính toán. 
Bên cạnh đó, ta cần tín hiệu calib cấp vào chân $V_{ref}$ của INA333 để hiệu chỉnh cho các giá trị không lý tưởng của OPAMP và cân bằng lại loadcell.

\section{Tính toán mạch V-I Converter}

Sau khi khuếch đại tín hiệu từ INA333, ta cần chuyển đổi tín hiệu từ 0-5V thành 4-20mA.
\begin{figure}[H]
    \centering
    \includegraphics[width=0.4\textwidth]{image/V_I_graph.png}
    \caption{Chuyển đổi V-I}
    \label{fig:V_I_graph}
\end{figure}

Ta có: $I = 3.2V + 4 \rightarrow I = \dfrac{V+1.25}{0.3125}$

\noindent
Cần mạch cộng điện áp với $V_{ref} = 1.25V$ và mạch chuyển đổi V-I.
\begin{figure}[H]
    \centering
    \includegraphics[width=0.85\textwidth]{image/V_I_circuit.png}
    \caption{Mạch cộng điện áp và V-I converter}
    \label{fig:V_I_circuit}
\end{figure}
\noindent
Ta có: $V_{o1} = \left(1 + \dfrac{R_2}{R_1}\right)\dfrac{V + 1.25}{2} \rightarrow \dfrac{R_2}{R_1} = 1$\\
Chọn $R_1 = R_2 = 10k\Omega$, $R_L =50 \Omega$ $R_3 = 0.3125k \Omega$.\\

\noindent
Mô phỏng trên NI multisim:
\begin{figure}[H]
    \centering
    \includegraphics[width=0.8\textwidth]{image/V_I_circuit_simu_m=1.png}
    \caption{Mô phỏng V-I converter với khối lượng 1kg}
    \label{fig:V_I_circuit_simu_m=1}
\end{figure}

\begin{figure}[H]
    \centering
    \includegraphics[width=0.8\textwidth]{image/V_I_circuit_simu_m=0.png}
    \caption{Mô phỏng V-I converter với khối lượng 0kg}
    \label{fig:V_I_circuit_simu_m=0}
\end{figure}

\textbf{Nhận xét:} Với khối lượng là 1kg, dòng ngõ ra là 20mA đúng với tính toán.
Tuy nhiên với khối lượng là 0kg, dòng ngõ ra đo được là 3.9mA lệch nhỏ so với lý thuyết 4mA do ảnh hưởng từ các thông số không lý tưởng của OPAMP và bởi vì không có trở với giá trị $0.3125k \Omega$ so với lý thuyết nên nhóm chọn nối tiếp 3 điện trở 100$\Omega$ và 1 điện trở 10$\Omega$.
Vì thế để đạt được kết quả này cần phải điều chỉnh $V_{calib}$.

\section{Thiết kế mạch đọc dòng}

Đo dòng 4-20mA bằng cách chuyển đổi dòng thành áp qua một điện trở $R_{shunt}$.
Sau đó đưa áp trên trở vào bộ khuếch đại vi sai để ra áp tương ứng.
\begin{figure}[H]
    \centering
    \includegraphics[width=0.5\textwidth]{image/current_measurement.png}
    \caption{Mạch đọc dòng}
    \label{fig:current_measurement}
\end{figure}

Chọn $R_1=R_2=32k\Omega$ để độ lợi khuếch đại vi sai $G=\dfrac{1.6M\Omega}{32k\Omega}=50$ và với $I=20mA$ thì $V_{out}=3.3V$.
Từ đó ta tính được điện trở $R_{shunt} = \dfrac{V_{shunt}}{I} = \dfrac{V_{out}}{G \times I} = \dfrac{3.3V}{50 \times 20mA} = 3.3 \Omega$.\\

Mô phỏng trên NI multisim:

\begin{figure}[H]
    \centering
    \includegraphics[width=0.8\textwidth]{image/I_V_m=1.png}
    \caption{Mô phỏng V-I converter với khối lượng 1kg}
    \label{fig:I_V_circuit_simu_m=1}
\end{figure}

\begin{figure}[H]
    \centering
    \includegraphics[width=0.8\textwidth]{image/I_V_m=0.png}
    \caption{Mô phỏng V-I converter với khối lượng 0kg}
    \label{fig:I_V_circuit_simu_m=0}
\end{figure}

\textbf{Nhận xét:} Điện áp sau khi đọc dòng sẽ được đệm qua 1 buffer để đưa vào khối ADC.
Với khối lượng là 1kg, điện áp ngõ ra là 3.3V đúng với tính toán.
Với khối lượng là 0kg, điện áp ngõ ra đo được là 0.65mV.
