\chapter{Tính toán mô phỏng}
\section{Mô hình loadcell}

Mô hình load cell được minh họa trong Hình~\ref{fig:loadcell_model}, ta có được $R_{in}$ và $R_{out}$ như sau:

\begin{figure}[H]
    \centering
    \begin{minipage}{0.35\textwidth}
        \centering
        \includegraphics[width=\textwidth]{image/Rin.png}
        \caption{$R_{in}$}
        \label{fig:rin}
    \end{minipage}
    \hfill
    \begin{minipage}{0.35\textwidth}
        \centering
        \includegraphics[width=\textwidth]{image/Rout.png}
        \caption{$R_{out}$}
        \label{fig:rout}
    \end{minipage}
\end{figure}

\begin{figure}[H]
    \centering
    \includegraphics[width=0.25\textwidth]{image/Loadcell_model_2.png}
    \caption{Mô hình Load cell mô phỏng}
    \label{fig:loadcell_model_2}
\end{figure}
\noindent
Ta có:
\[
\left\{
\begin{aligned}
R_{in} &= \dfrac{(R_0+R_f) + (R_0-R_f)}{2} + 2R_s \\
R_{out} &= R_0 - \dfrac{R_f^2}{R_0 + 2R_s}
\end{aligned}
\right.
\]
\noindent
Khi chưa có tải $m = 0$, cầu cân bằng:
\[
\rightarrow
\left\{
\begin{aligned}
R_f &= 0 \\[4pt]
R_0 &= R_{out} = 1000\,\Omega \\[4pt]
R_s &= 33\,\Omega
\end{aligned}
\right.
\]
\noindent
Khi có tải $m \neq 0 \rightarrow R_f \neq 0$. Ta có:
\[
\left\{
\begin{aligned}
I_2 &= I_3 = \dfrac{I_1}{2} \\
V_{out} &= V_a - V_b = (I_1R_s+I_3(R_0+R_f)) -(I_1R_s+I_2(R_0-R_f)) \\
\end{aligned}
\right.
\]
\begin{align*}
&\rightarrow\; V_{\text{out}} 
= \left[ I_1 R_s + \dfrac{I_1}{2}(R_0 + R_f) \right] 
- \left[ I_1 R_s + \dfrac{I_1}{2}(R_0 - R_f) \right] 
= I_1 R_f 
= R_f \dfrac{V_{\text{in}}}{R_{\text{in}}} \\
%
&\rightarrow\; R_f 
= R_{\text{in}} \dfrac{V_{\text{out}}}{V_{\text{in}}} 
= R_{\text{in}} \times V_{RO} \times \dfrac{\text{m}}{1\,\text{kg}} 
= 1066 \times 1 \times 10^{-3} \times m \\
%
&\boxed{\rightarrow\; R_f = 1.066 \times m \quad (\Omega)}
\end{align*}

Mô phỏng trên NI multisim:
%Cho hình vào đây
\begin{figure}[H]
    \centering
    \begin{minipage}{0.4\textwidth}
        \centering
        \includegraphics[width=\textwidth]{image/m=0.png}
        \caption{Load cell khi $m = 0$ kg}
        \label{fig:m=0}
    \end{minipage}
    \hfill
    \begin{minipage}{0.4\textwidth}
        \centering
        \includegraphics[width=\textwidth]{image/m=1.png}
        \caption{Load cell khi $m = 1$ kg}
        \label{fig:m=1}
    \end{minipage}
\end{figure}

\textbf{Nhận xét:} Tín hiệu ngõ ra của load cell mô phỏng đã đúng với các tính toán.

Như đã đề cập ở chương 2, ngõ ra cần khuếch đại lên 5V. Chính vì thế độ lợi là 1000. 

Ta có độ lợi của INA333 $G = 1+\frac{100k\Omega}{R_G} \rightarrow R_G = 100 \Omega$

\begin{figure}[H]
    \centering
    \includegraphics[width=0.9\textwidth]{image/loadcell_ampli.png}
    \caption{Load cell sau khi khuếch đại}
    \label{fig:loadcell_amplifier}
\end{figure}

\textbf{Nhận xét:} Tín hiệu ngõ ra của load cell sau khi khuếch đại đã giống với tính toán. 
Bên cạnh đó, ta cần tín hiệu calib cấp vào chân $V_{ref}$ của INA333 để hiệu chỉnh cho các giá trị không lý tưởng của OPAMP và cân bằng lại loadcell.

\section{Tính toán mạch V-I Converter}

Sau khi khuếch đại tín hiệu từ INA333, ta cần chuyển đổi tín hiệu từ 0-5V thành 4-20mA.
\begin{figure}[H]
    \centering
    \includegraphics[width=0.4\textwidth]{image/V_I_graph.png}
    \caption{Chuyển đổi V-I}
    \label{fig:V_I_graph}
\end{figure}

Ta có: $I = 3.2V + 4 \rightarrow I = \dfrac{V+1.25}{0.3125}$

\noindent
Cần mạch cộng điện áp với $V_{ref} = 1.25V$ và mạch chuyển đổi V-I.
\begin{figure}[H]
    \centering
    \includegraphics[width=0.85\textwidth]{image/V_I_circuit.png}
    \caption{Mạch cộng điện áp và V-I converter}
    \label{fig:V_I_circuit}
\end{figure}
\noindent
Ta có: $V_{o1} = \left(1 + \dfrac{R_2}{R_1}\right)\dfrac{V + 1.25}{2} \rightarrow \dfrac{R_2}{R_1} = 1$\\
Chọn $R_1 = R_2 = 10k\Omega$, $R_L =50 \Omega$ $R_3 = 0.3125k \Omega$.\\

\noindent
Mô phỏng trên NI multisim:
\begin{figure}[H]
    \centering
    \includegraphics[width=0.8\textwidth]{image/V_I_circuit_simu_m=1.png}
    \caption{Mô phỏng V-I converter với khối lượng 1kg}
    \label{fig:V_I_circuit_simu_m=1}
\end{figure}

\begin{figure}[H]
    \centering
    \includegraphics[width=0.8\textwidth]{image/V_I_circuit_simu_m=0.png}
    \caption{Mô phỏng V-I converter với khối lượng 0kg}
    \label{fig:V_I_circuit_simu_m=0}
\end{figure}

\textbf{Nhận xét:} Với khối lượng là 1kg, dòng ngõ ra là 20mA đúng với tính toán.
Tuy nhiên với khối lượng là 0kg, dòng ngõ ra đo được là 3.9mA lệch nhỏ so với lý thuyết 4mA do ảnh hưởng từ các thông số không lý tưởng của OPAMP và bởi vì không có trở với giá trị $0.3125k \Omega$ so với lý thuyết nên nhóm chọn nối tiếp 3 điện trở 100$\Omega$ và 1 điện trở 10$\Omega$.
Vì thế để đạt được kết quả này cần phải điều chỉnh $V_{calib}$.

\section{Thiết kế mạch đọc dòng}

Đo dòng 4-20mA bằng cách chuyển đổi dòng thành áp qua một điện trở $R_{shunt}$.
Sau đó đưa áp trên trở vào bộ khuếch đại vi sai để ra áp tương ứng.
\begin{figure}[H]
    \centering
    \includegraphics[width=0.5\textwidth]{image/current_measurement.png}
    \caption{Mạch đọc dòng}
    \label{fig:current_measurement}
\end{figure}

Chọn $R_1=R_2=32k\Omega$ để độ lợi khuếch đại vi sai $G=\dfrac{1.6M\Omega}{32k\Omega}=50$ và với $I=20mA$ thì $V_{out}=3.3V$.
Từ đó ta tính được điện trở $R_{shunt} = \dfrac{V_{shunt}}{I} = \dfrac{V_{out}}{G \times I} = \dfrac{3.3V}{50 \times 20mA} = 3.3 \Omega$.\\

Mô phỏng trên NI multisim:

\begin{figure}[H]
    \centering
    \includegraphics[width=0.8\textwidth]{image/I_V_m=1.png}
    \caption{Mô phỏng V-I converter với khối lượng 1kg}
    \label{fig:I_V_circuit_simu_m=1}
\end{figure}

\begin{figure}[H]
    \centering
    \includegraphics[width=0.8\textwidth]{image/I_V_m=0.png}
    \caption{Mô phỏng V-I converter với khối lượng 0kg}
    \label{fig:I_V_circuit_simu_m=0}
\end{figure}

\textbf{Nhận xét:} Điện áp sau khi đọc dòng sẽ được đệm qua 1 buffer để đưa vào khối ADC.
Với khối lượng là 1kg, điện áp ngõ ra là 3.3V đúng với tính toán.
Với khối lượng là 0kg, điện áp ngõ ra đo được là 0.65mV.

\section{Thiết kế mạch bù nhiệt}
\subsection{Vai trò của việc bù nhiệt}
Trên lý thuyết, cảm biến được thiết kế khi nó đang được tham chiếu trong một môi trường thiết kế nhất định, khi đó tính chính xác của cảm biến đó sẽ đúng nhưng chỉ đối với tại môi trường mà nó được sản xuất. Trong thực tế, cảm biến sẽ dùng ở những môi trường khác nhau, cụ thể với load cell khi xem xét về vấn đề nhiệt độ nó sẽ có những sai số theo như datasheet mô tả.

Các thông số của loadcell bị ảnh hưởng bởi nhiệt độ:\\
\begin{table}[htbp]
\begin{center}
\begin{tabular}{|c|c|}
\hline
Ảnh hưởng nhiệt độ tới độ nhạy \%RO/$^o C$& 0.003\\
\hline
Ảnh hưởng nhiệt độ tới điểm không \%RO/$^o C$ & 0.02              \\
\hline
\end{tabular}
\end{center}
\end{table}\\


Ảnh hưởng nhiệt độ tới độ nhạy: 
$0.003\%RO/^{\circ}\text{C}$ có nghĩa là: cứ mỗi $1^{\circ}\text{C}$ nhiệt độ môi trường thay đổi (tăng hoặc giảm) so với nhiệt độ tham chiếu (thường là $25^{\circ}\text{C}$), thì độ nhạy của Load Cell sẽ thay đổi $\mathbf{0.003\%}$ so với giá trị $RO$ danh định của nó.
Hậu quả: Nếu nhiệt độ tăng, ví dụ, $10^{\circ}\text{C}$ so với nhiệt độ chuẩn, thì độ nhạy sẽ thay đổi $10 \times 0.003\% = 0.03\%$. Điều này dẫn đến sự sai lệch trong phép đo tải. Ví dụ: khi bạn đặt tải tối đa, tín hiệu ngõ ra có thể không phải là $RO$ mà là $RO \pm 0.03\%$.

Ảnh hưởng nhiệt độ tới điểm không: là tín hiệu ngõ ra của Load Cell khi không có bất kỳ tải trọng nào tác dụng. Load Cell lý tưởng phải có ngõ ra bằng $0$ khi không tải. $0.02\%RO/^{\circ}\text{C}$ có nghĩa là: cứ mỗi $1^{\circ}\text{C}$ nhiệt độ môi trường thay đổi, thì tín hiệu điểm không của Load Cell sẽ thay đổi (lệch đi) $\mathbf{0.02\%}$ so với giá trị $RO$ danh định. Hậu quả: Sự thay đổi này làm cho điểm khởi đầu của phép đo bị sai lệch. Ví dụ, nếu nhiệt độ tăng $1^{\circ}\text{C}$, ngay cả khi không có tải, Load Cell vẫn xuất ra một tín hiệu tương đương $0.02\%$ của tải tối đa. Đây được gọi là trôi điểm không (Zero Drift).

Trong bài báo cáo trình bày việc dùng một RTD đo nhiệt độ để từ đó có thể tự bù khoảng trôi điểm 0 do nhiệt độ gây ra trên loadcell.

\subsection{Mô phỏng bù nhiệt và cân bằng}

\begin{figure}[H]
    \centering
    \includegraphics[width=0.8\textwidth]{image/RTD.jpg}
    \caption{RTD bù nhiệt}
    \label{fig:RTD}
\end{figure}

Hình~\ref{fig:RTD} sử dụng 1 mạch cầu wheatstone cân bằng để đọc cảm biến RTD, đặt biến RTD để mô tả phương trình thay đổi điện trở theo nhiệt độ. Phương trình thay đổi giá trị điện trở của RTD theo nhiệt độ

$$ \boxed{R = 100 + 0.390802T} $$

Để triệt tiêu tín hiệu offset trôi điểm 0 do sai số nhiệt độ gây ra, phải bù bằng đúng mức điện áp khi đọc cảm biến về.
\begin{flalign*}
&V_m = V_m(+) - V_m(-) = \frac{E_s R_T}{R_T + R_{26}} - \frac{E_s R_{28}}{R_{28} + R_{29}} & \\
&\text{Có } R_T = 100 + 0.390802T & \\
&\implies V_m = E_s \left( \frac{100 + 0.390802T}{100 + 0.390802T + R_{26}} - \frac{R_{28}}{R_{28} + R_{29}} \right) & \\
&\text{Chọn } R_{28} = 100, R_{26} = R_{29} \text{ và } R_{26} + 100 \gg 0.390802T & \\
&\text{Điều kiện để bù V diff offset là ta phải có Vm = $ \%RO/^\circ\text{C} $}\\
&\implies V_m = E_s \left( \frac{100 + 0.390802T}{100 + R_{26}} - \frac{100}{100 + R_{26}} \right) & \\
&\iff V_m = E_s \frac{0.390802T}{100 + R_{26}} = 5\text{V} \times 0.02\%T & \\
&\text{Chọn } E_s = 5\text{V} \implies R_{26} = 1854\Omega &
\end{flalign*}

Sử dụng INA333 để đọc tín hiệu và ngõ ra của cảm biến nhiệt độ và đưa vào bộ cộng để kết hợp với việc tinh chỉnh thủ công từ bộ cộng, từ đó toàn bộ giá trị sẽ được đưa vào chân Vref (chân 5) của INA333 đọc loadcell, Hình~\ref{fig:calib}

\begin{figure}[H]
    \centering
    \includegraphics[width=0.8\textwidth]{image/calib.jpg}
    \caption{Mạch cộng bù nhiệt và cân bằng thủ công}
    \label{fig:calib}
\end{figure}

Sử dụng OP07CS8 để làm mạch cộng tín hiệu bù nhiệt độ và tín hiệu cân chỉnh offset thủ công. Tín hiệu bù nhiệt được đưa vào chân âm để có ngõ ra là một điện áp âm để bù với offset nhiệt độ trên loadcell, chân cộng được kết nối với biến trở để điều chỉnh thủ công. Chọn $R_{24} = R_{25} = 10k$ để độ lợi khuếch đại là 1.
\clearpage 
Sử dụng một nguồn DC với biến Vtemp để giả sử độ trôi của loadcell, Hình~\ref{fig:vtemp50} và Hình~\ref{fig:vtemp100} đang xét tại 2 mốc nhiệt độ là $50^\circ\text{C}$ và $100^\circ\text{C}$ để xem xét sự trôi.

Theo công thức được cung cấp thì Vtemp = $0.02\% \times 5\text{mV} \times T = 1\mu \times T$
\begin{figure}[H]
    \centering
    \includegraphics[width=0.8\textwidth]{image/vtemp50.jpg}
    \caption{Tại nhiệt độ $50^\circ\text{C}$}
    \label{fig:vtemp50}
\end{figure}

\begin{figure}[H]
    \centering
    \includegraphics[width=0.8\textwidth]{image/vtemp100.jpg}
    \caption{Tại nhiệt độ $100^\circ\text{C}$}
    \label{fig:vtemp100}
\end{figure}

Khảo sát giá trị I và V ngõ ra đọc được tại $0^\circ\text{C}$ bằng Hình~\ref{fig:at0_I}, Hình~\ref{fig:at0_V}  và $50^\circ\text{C}$ bằng Hình~\ref{fig:at50_I}, Hình~\ref{fig:at50_V}, có thể thấy không có sự chênh lệch lớn giữa 2 kết quả đo cho thấy mạch bù nhiệt khi mô phỏng hoạt động hiệu quả.

\begin{figure}[H]
    \centering
    \includegraphics[width=0.8\textwidth]{image/at0_I.jpg}
    \caption{Tại nhiệt độ $0^\circ\text{C}$, quan sát I}
    \label{fig:at0_I}
\end{figure}

\begin{figure}[H]
    \centering
    \includegraphics[width=0.8\textwidth]{image/at0_V.jpg}
    \caption{Tại nhiệt độ $0^\circ\text{C}$, quan sát V}
    \label{fig:at0_V}
\end{figure}

\begin{figure}[H]
    \centering
    \includegraphics[width=0.8\textwidth]{image/at50_I.jpg}
    \caption{Tại nhiệt độ $50^\circ\text{C}$, quan sát I}
    \label{fig:at50_I}
\end{figure}

\begin{figure}[H]
    \centering
    \includegraphics[width=0.8\textwidth]{image/at50_V.jpg}
    \caption{Tại nhiệt độ $50^\circ\text{C}$, quan sát V}
    \label{fig:at50_V}
\end{figure}
