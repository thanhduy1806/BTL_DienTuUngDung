\chapter{Kết luận}

Nhóm đã hoàn thành việc thiết kế mạch đo khối lượng (tiểu ly) sử dụng Load cell YZC-133, đáp ứng đầy đủ các yêu cầu kỹ thuật của đề tài đưa ra, bao gồm tầm đo 0-1kg và độ phân giải 2mg. Nhóm đã trình bày một quy trình thiết kế mạch từ khâu lựa chọn linh kiện, tính toán lý thuyết, mô phỏng các khối chức năng, thiết kế bộ chuyển đổi ADC, đến việc thiết kế và thi công phần cứng. Các giá trị tính gần đúng với khi tính lý thuyết. Mạch được thiết kế khá hoàn chỉnh từ ngõ ra Load cell (tín hiệu mV nhỏ) qua bộ khuếch đại vi sai INA333 , mạch chuyển đổi V-I (0-5V thành 4-20mA), thiết kế mạch bù nhiệt, mạch đọc dòng, và cuối cùng là mạch đọc ADC. Các kết quả mô phỏng cho thấy giá trị đầu ra (dòng điện và điện áp) gần đúng với giá trị lý thuyết. \\

Kết quả mô phỏng gần đúng với tính toán lý thuyết đưa ra,  sai số nhỏ do ảnh hưởng từ các thông số không lý tưởng của OPAMP và sai số nội tại của các khối mô phỏng. Khi mô phỏng bộ ADC bằng phần mềm NI Multisim thì các sai số nội tại bên trong mỗi bộ cộng cũng góp phần làm sai lệch đến kết quả tại ngõ ra nhưng vẫn có thể chấp nhận được.\\

Về phần cứng, nhóm đã thực hiện đầy đủ các bước từ việc chọn linh kiện có sẵn, lên thiết kế sơ đồ nguyên lí các khối nguồn, khối chức năng, layout PCB, thực hiện đặt thi công board mạch và hàn các linh kiện lên board. Tuy nhiên trong quá trình thiết kế board, nhóm có một số sai sót dẫn đến board sau khi đặt in không hoạt động được như mong muốn, nhưng nhóm đã tìm được giải pháp và khắc phục được vấn đề đó, các kết quả đo đạc thực tế có giá trị gần chính xác với mô phỏng và tính toán nhưng có sai số khá nhỏ, khoảng 0.04mA. \\ 

Tóm lại, nhóm đã thiết kế mô phỏng và thực hiện phần cứng thực tế một mạch đo khối lượng có độ chính xác và độ phân giải cao gần với đề tài yêu cầu. Các giá trị đo đạc được cũng thể hiện rõ ràng và tương đối chính xác. Tuy nhiên, kết quả đối chiếu giữa mô phỏng phần mềm và phần cứng thực tế vẫn có sai số khá nhỏ có thể là do điều kiện môi trường thực tế, tính chất linh kiện thực, nhưng vẫn có thể tạm chấp nhận được. Giải pháp khắc phục có thể bằng phần cứng và phần mềm, nhóm sẽ nghiên cứu và cập nhật trong tương lai